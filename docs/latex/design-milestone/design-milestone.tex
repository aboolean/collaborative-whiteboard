%%%%%%%%%%%%%%%%%%%%%%%%%%%%%%%%%%%%%%%%%
% Journal Article
% LaTeX Template
% Version 1.3 (9/9/13)
%
% This template has been downloaded from:
% http://www.LaTeXTemplates.com
%
% Original author:
% Frits Wenneker (http://www.howtotex.com)
%
% License:
% CC BY-NC-SA 3.0 (http://creativecommons.org/licenses/by-nc-sa/3.0/)
%
%%%%%%%%%%%%%%%%%%%%%%%%%%%%%%%%%%%%%%%%%

%----------------------------------------------------------------------------------------
%    PACKAGES AND OTHER DOCUMENT CONFIGURATIONS
%----------------------------------------------------------------------------------------

\documentclass[twoside]{article}

\usepackage{lipsum}
\usepackage{graphicx}

\usepackage{listings}
\usepackage{color}

\lstset{
    language=Java,                % choose the language of the code
    basicstyle=\footnotesize,       % the size of the fonts that are used for the code
    numbers=left,                   % where to put the line-numbers
    numberstyle=\footnotesize,      % the size of the fonts that are used for the line-numbers
    stepnumber=1,                   % the step between two line-numbers. If it is 1 each line will be numbered
    numbersep=5pt,                  % how far the line-numbers are from the code
    backgroundcolor=\color{white},  % choose the background color. You must add \usepackage{color}
    showspaces=false,               % show spaces adding particular underscores
    showstringspaces=false,         % underline spaces within strings
    showtabs=false,                 % show tabs within strings adding particular underscores
    frame=single,           % adds a frame around the code
    tabsize=2,          % sets default tabsize to 2 spaces
    captionpos=b,           % sets the caption-position to bottom
    breaklines=true,        % sets automatic line breaking
    breakatwhitespace=false,    % sets if automatic breaks should only happen at whitespace
    escapeinside={\%*}{*)}          % if you want to add a comment within your code
}

\usepackage{courier}
\usepackage[sc]{mathpazo} % Use the Palatino font
\usepackage[T1]{fontenc} % Use 8-bit encoding that has 256 glyphs
\linespread{1.05} % Line spacing - Palatino needs more space between lines
\usepackage{microtype} % Slightly tweak font spacing for aesthetics

\usepackage[hmarginratio=1:1,top=32mm,columnsep=20pt]{geometry} % Document margins
\usepackage{multicol} % Used for the two-column layout of the document
\usepackage[hang, small,labelfont=bf,up,textfont=it,up]{caption} % Custom captions under/above floats in tables or figures
\usepackage{booktabs} % Horizontal rules in tables
\usepackage{float} % Required for tables and figures in the multi-column environment - they need to be placed in specific locations with the [H] (e.g. \begin{table}[H])
\usepackage{hyperref} % For hyperlinks in the PDF

\usepackage{lettrine} % The lettrine is the first enlarged letter at the beginning of the text
\usepackage{paralist} % Used for the compactitem environment which makes bullet points with less space between them

\usepackage{abstract} % Allows abstract customization
\renewcommand{\abstractnamefont}{\normalfont\bfseries} % Set the "Abstract" text to bold
\renewcommand{\abstracttextfont}{\normalfont\small\itshape} % Set the abstract itself to small italic text

\usepackage{titlesec} % Allows customization of titles
\renewcommand\thesection{\Roman{section}} % Roman numerals for the sections
\renewcommand\thesubsection{\Roman{subsection}} % Roman numerals for subsections
\titleformat{\section}[block]{\large\scshape}{\thesection.}{1em}{} % Change the look of the section titles
\titleformat{\subsection}[block]{\large}{\thesubsection.}{1em}{} % Change the look of the section titles

\usepackage{fancyhdr} % Headers and footers
\pagestyle{fancy} % All pages have headers and footers
\fancyhead{} % Blank out the default header
\fancyfoot{} % Blank out the default footer
\fancyhead[C]{Design Milestone $\bullet$ 6.005 Project 2 $\bullet$ 2013 Dec 3} % Custom header text
\fancyfoot[RO,LE]{\thepage} % Custom footer text

%----------------------------------------------------------------------------------------
%    TITLE SECTION
%----------------------------------------------------------------------------------------

\title{\vspace{-15mm}\fontsize{24pt}{10pt}\selectfont\textbf{Design Milestone}} % Article title

\author{
\large
\textsc{Andre Aboulian, Cathleen Gendron, \& Jon Beaulieu}\\[2mm] % member name
\normalsize 6.005 Software Construction - Fall 2013 - Project 2: "Collaborative Whiteboard"
\vspace{-5mm}
}
\date{}

%----------------------------------------------------------------------------------------

\begin{document}

\maketitle % Insert title

\thispagestyle{fancy} % All pages have headers and footers

\hspace{10mm}

\section{User Functionality Overview}

\subsection{Components}

\includegraphics[keepaspectratio=1,width=6in]{img/gui-sketch.jpg}

\subsubsection{Selector}

The board selector in the left pane includes a list of all current whiteboards and appears the same for all users. Each line represents an individual board, which are numbered sequentially and named by the user. Upon clicking the "+" button, the user will be prompted to name the new board. When the board has been created on the server, it will be appended to the list for each user. Selecting a board in the list will download the board from the server, overwrite the local copy if one exists, and display the board in the canvas window. 

\subsubsection{Board Editors}

Displays a list of users, including the viewer, who are currently modifying the selected board. This list will be updated as users enter and exit the board.

\subsubsection{Thickness Selector}

This tool allows the user to select a brush/eraser thickness for drawing on the whiteboard.

\subsubsection{Color Selector}

The main color palette displays a grid of colors from which the user can choose to paint with. The color currently in use will be highlighted. Clicking the "more colors" button will open Swing's built-in color chooser, which will offer a larger selection of colors.

\subsubsection{Erasing Tools}

The erase button will allow the user to toggle between erasing and painting. "Erasing" will be defined as drawing with a white selection. Erasing will happen in the same order as drawing, so whichever request reaches the server first will erase all that has been drawn under it. Toggling back to painting will restore the user's previous color choice.

\subsubsection{Whiteboard Window}

Displays the currently selected whiteboard, including all of its drawn strokes and erasures. The whiteboard be real-time interactive to allow users to collaborate simultaneously. Edits will be made in the order that modifications reach the server. In other words, a stroke logged on the server at a specific instant will be drawn over any strokes drawn before that instant.

\subsection{Behavior}

\paragraph{Erasing} TEXT

\paragraph{Editing a Deleted Board} TEXT

\section{Server-Client Communication}

\subsection{Protocol}

\subsubsection{Grammar}
The following grammar will facilitate the text-based communication between the clients and the server. The server will send \texttt{StoC\_MSG} messages to the client, which will be able to send \texttt{CtoS\_MSG} messages back to the server.

\vspace{5mm}

\setlength{\parindent}{0in}

\texttt{StoC\_MSG :== (STROKE | BRD\_INFO | BRD\_DEL | USER\_INIT | BRD\_USERS) N}\\

\texttt{CtoS\_MSG :== (STROKE | SEL | BRD\_REQ | BRD\_DEL | BRD\_ALL | USER\_REQ) N}\\


\texttt{STROKE :== "stroke" S BOARD\_ID S THICK S COORDS S COLOR}\\
\texttt{COORDS :== X1 S Y1 S X2 S Y2}\\
\texttt{X1, Y1, X2, Y2 :== INT}\\
\texttt{COLOR :== [0-255] S [0-255] S [0-255]}\\
\texttt{THICK :== [1-10]}\\

\texttt{SEL :== "select" S BOARD\_ID}\\

\texttt{BRD\_REQ :== "board\_req" S NAME}\\
\texttt{BRD\_ALL :== "board\_all"}\\
\texttt{BRD\_INFO :== "board" S BOARD\_ID S NAME}\\
\texttt{BRD\_DEL :== "del" S BOARD\_ID}\\
\texttt{BRD\_USERS :== "board\_users" S BOARD\_ID (S USER\_NAME)+}\\

\texttt{USER\_REQ :== "user\_req" S USER\_NAME}\\
\texttt{USER\_INIT :== "you\_are" S USER\_NAME}\\

\texttt{NAME :== [$\char`\^$N]}\\
\texttt{USER\_NAME :== [A-Za-z]([A-Za-z0-9]?)+}\\
\texttt{BOARD\_ID :== INT}\\

\texttt{INT :== [0-9]+}\\
\texttt{N :== "$\backslash$r?$\backslash$n"}\\
\texttt{S :== " "}\\

\setlength{\parindent}{15pt} %default


\subsubsection{Usage}

\paragraph{Adding Users} Upon entering a username in the client application, a \texttt{USER\_REQ} message will be sent to the server to request the desired username. (Note that regex checking for \texttt{USER\_NAME} occurs on the client side before this request is made.) The server responds with a \texttt{USER\_INIT} message, signifying the acquired username for the client. If there is a username conflict, one is chosen for the client. When the client is ready to accept information about existing boards, it calls \texttt{BRD\_ALL} to begin receiving \texttt{BRD\_INFO} messages for all previously created boards.

\paragraph{Adding Boards} When a client wants to create a new whiteboard, it sends a \texttt{BRD\_REQ} request to the server with a desired \texttt{NAME} (duplicate names allowed). Once the server has initialized a new internal board object, it sends a \texttt{BRD\_INFO} message to all connected users to inform them of the newly available board. Note that the \texttt{BOARD\_ID} used is a number unique to each whiteboard and is never reused. This is a different number than the sequential board numbering in the GUI, although the order is preserved.

\paragraph{Removing Boards} Deleting boards entails a process similar to adding them. A client sends a \texttt{BRD\_DEL} request to the server, which forwards this requests to all other users. The server internally disassociates all connected users and removes the board, taking care to ignore drawing requests and selection requests for this deleted board.

\paragraph{Selecting Boards} Upon selecting a different board, the client sends a \texttt{SEL} request to the server. The server clears all stroke messages queued to update the client's whiteboard before associating the requested whiteboard object to the user, if available. The \texttt{SEL} command also requests all previously drawn strokes to be sent to the client. A \texttt{BRD\_USERS} messages is sent to all users of the previous and current whiteboard to inform them of this change in editors.

\paragraph{Disconnecting Users}
When a client disconnects -- by either severing the connection or closing the client application -- the server closes the associated socket and streams, disassociates the user from its whiteboard, and removes the user from its main users list. A \texttt{BRD\_USERS} message is sent to all users of the board that was being edited.

\paragraph{Drawing Strokes} When a client draws a stroke in the selected whiteboard, a sequence of \texttt{STROKE} messages are sent to the server. The corresponding lines are logged as drawn in the order they are received by the server. The server proceeds to forward the \texttt{STROKE} messages to all users editing the same whiteboard, including the user who made the edit. Since the server sends the \texttt{STROKE} updates in the order they were made, the user's stroke will be covered by the echoed version from the server. This will not be visually apparent and will ensure that concurrently drawn lines appear constant across clients.

\subsection{Data Transport}

\subsubsection{New Connections}

The Whiteboard server will maintain a background thread that listens for new connections. Upon accepting a new socket, a new thread prompts the client for a username, and then instantiates a new User, which is then added to the ArrayList of all Users.

\subsubsection{Request Streams}

GUI: The GUI maintains two threads, one which listens for requests from the associated User, and another which contains a BlockingQueue containing requests to be sent to the User. Possible requests include creating a new Whiteboard, deleting a Whiteboard, selecting a different Whiteboard, or sending a Stroke.

User: Each User object holds the client-side socket. The User maintains two threads, one which listens for requests from either its MasterBoard or its GUI, and another which contains a BlockingQueue containing requests to be sent to the GUI. The request() method parses GUI requests and sends the appropriately formatted text protocol message to either the MasterBoard (when a Stroke is made) or the WhiteboardServer (for other User/Whiteboard requests).

MasterBoard: The MasterBoard maintains two threads, one which listens for requests from any of its associated Users, and another which contains a BlockingQueue containing requests to be sent to all of its associated Users. For example, the listening thread will accept a request to create a new Stroke to be added to its strokes ArrayList, and then send that updated ArrayList to all its Users.

WhiteboardServer: The WhiteboardServer is responsible for instantiating new Users with each new connection, and creating new MasterBoard objects in response to requests from Users. It maintains an InputStream to receive requests from the Users, but does not need an OutputStream.

\section{Threads and Queues}

\paragraph{Client} The GUI maintains one \texttt{SwingWorker} thread, which reads and processes requests received from the server. Possible messages include notifications of a new Whiteboard, a deleted Whiteboard, or a new Stroke. Note that output is written directly in the client application in the main thread.

\paragraph{User} Each \texttt{User} object on the server object holds a socket connected to an individual client. The \texttt{User} class has two outgoing message queues: one is dedicated to \texttt{STROKE} messages, whereas the other sends general messages. The outgoing stroke queue can be cleared upon changing boards without affecting other messages that need to be sent to the client. The \texttt{User} maintains two threads, one which receives and handles requests from the client, and another which sends messages to the client. The latter thread consumes from both the message queue and the stroke queue, prioritizing sending general messages over stroke updates.

\paragraph{MasterBoard} Each time a stroke is written to a \texttt{MasterBoard}, the \texttt{WhiteLine} passed is queued for later processing. Each \texttt{MasterBoard} has one thread which consumes lines from this queue, makes the change, and sends the update to all users.

\paragraph{WhiteboardServer} The \texttt{WhiteboardServer} has a thread continuously running to accepts new users that connect to its server-side socket. Each time a user connects, the corresponding \texttt{Socket} instance is passed to a new dedicated thread wile the original thread continues to accept new users. In these individual threads, the username handshake is resolved and a new \texttt{User} instance is created. In other words, there is one thread always running and there are as many additional threads as there are clients being instantiated. Each additional thread is closed once the user has been created.

\section{Thread Safety}

\subsection{Processes}

\subsubsection{Adding New Users}

Upon receiving a \texttt{USER\_REQ} message, the WhiteboardServer calls \texttt{createNewUser} on the main server thread. This instantiates the new User object and adds it to the \texttt{users} ArrayList. This method locks on the \texttt{boards} field first, and then the \texttt{users} field. This ensures that only one User is created at a time, and that behavior that is dependent other User objects, such as username and ID assignment, is consistent.

\subsubsection{Adding and Removing Boards}

Upon receiving a \texttt{BRD\_REQ} message, the WhiteboardServer calls \texttt{createBoard}, which instantiates the new MasterBoard object and adds it to the \texttt{boards} ArrayList. This method locks on the \texttt{boards} field first, and then the \texttt{users} field. If multiple requests for a new board are sent, the locks ensure that another board is not created until the current board is created, added to the ArrayList, and all Users are updated with these changes.

Upon receiving a \texttt{BRD\_DEL} message, the WhiteboardServer calls \texttt{removeBoard}. This removes the board from the server's \texttt{boards} ArrayList, as well disassociating all Users currently active on that board. This method locks on the \texttt{boards} field and then the \texttt{users}, so that only one board can be deleted at a time, and so that one thread cannot be deleting a board from the ArrayList while another is attempting to add one.

\subsubsection{Drawing Strokes}

Each time a line is drawn on a client's Canvas, a request is sent from the WhiteboardClient to the MasterBoard via a buffer. The MasterBoard puts this message on a queue of pending requests, then locks on the \texttt{strokes} ArrayList and appends the stroke encoded in the message. The use of the threadsafe BlockingQueue ensures that all strokes are sent from the GUI to the MasterBoard in the correct order. Before releasing the lock, all Users associated with this board are informed of the newly added stroke.

\subsubsection{Selecting Boards}

When the client selects a new board, the server fetches the board to be sent. The User calls \texttt{allStrokes} to obtain all of the strokes associated with the new board. This method locks on the \texttt{strokes} ArrayList, so that no new strokes can be added before the User has received the entire ArrayList. The User also adds itself to the MasterBoard's \texttt{users} ArrayList, locking on that field to ensure that only one User is added at a time.

\subsection{Averted Race Conditions}

\subsubsection{New Board and New User}

Both \texttt{BRD\_REQ} and \texttt{USER\_REQ} messages are handled on the main server thread. Since only one of these requests can be processed at a time, there cannot be any racing between the two methods that would potentially cause a newly created board to be lost for some Users. 

\subsubsection{Remove Board and New User}

Similar to the strategy in the previous section, both \texttt{DEL\_BRD} and \texttt{USER\_REQ} requests are handled on the main server thread. This eliminates any racing in which a new User would be created and still receive a board that had actually been deleted.

\subsubsection{Concurrent Strokes}

Because the GUI sends the new stroke messages through a BlockingQueue, the order is preserved. The MasterBoard then appends the strokes to its \texttt{strokes} in the order that they are received from the clients. (Note that this is not necessarily exactly the same as the order in which each stroke is sent from each GUI, but every User will receive strokes in the exact same order when receiving updates from the MasterBoard.) Because the \texttt{makeStrokes} method locks on \texttt{strokes} while it processes a request, only one stroke will be processed at a time, and the risk of losing strokes or having clients receive stroke updates in different orders is eliminated. Similarly, while a newly selected board is being transferred to a client, no new strokes can be added, to eliminate these same risks.

\subsubsection{Atomic ID Generation}

All object IDs are generated by a static AtomicInteger associated with each class. Because the AtomicInteger is threadsafe, any risks associated with interleaving integer operations are eliminated.

\subsection{Thread-Safe Collections}

The server-side classes \texttt{User} and \texttt{MasterBoard} contain a number of ArrayList fields, referenced in the preceding sections. With the exception of \texttt{strokes} in \texttt{MasterBoard}, all of these fields will utilize the thread-safe synchronized wrapper, to add additional protection against concurrency problems, such as deadlock. The \texttt{strokes} field will not use this wrapper, because the \texttt{MasterBoard} uses a BlockingQueue for pending changes, which ensures that the \texttt{strokes} ArrayList is not accessed in multiple threads concurrently, and to avoid the synchronized wrapper's performance penalty on an object accessed so frequently.

\section{Testing}

\subsection{Constructor Testing}

\subsubsection{New Board} 

\paragraph{Valid Name}
Test the creation of a new whiteboard with a valid name. This is simple to conduct by simply clicking the "+" button under the list of whiteboards. The client should be prompted to enter a name. Enter a valid name that is not already in use. The server should create a new \texttt{MasterBoard} object and add it to the list of existing boards.  This should be seen by all clients. The new whiteboard should be completely white.

\paragraph{Invalid Name}
Test the creation of a new whiteboard with an invalid name. Again, conduct by simply clicking the "+" button under the list of whiteboards. The client should be prompted to enter a name. Enter a name that is already in use. The server should inform the user that the name is already in use, then create a new \texttt{MasterBoard} object with a default name assigned by the server. Again, this new \texttt{MasterBoard} should be added to the list of existing boards, and should be seen by all users. The new whiteboard should be completely white.

\subsubsection{New User}

\paragraph{Valid Name}
Connect to the whiteboard server.  When prompted for a username, enter a valid one - that is, one that is not already in use.  The server should accept this name and assign it to the user, then allow the user to select a board and begin drawing.

\paragraph{Invalid Name}
Connect to the whiteboard server. When prompted for a username, enter a name that is already in use. The server should notify the new client that his/her choice of username is already in use, and should assign the new user a different (valid) name from a predefined list of names.  The new client should then be allowed to select a board and begin drawing.

\subsection{Normal Functionality}

\subsubsection{Loading a Board}
Tests behavior when a user attempts to load an existing board to work on.  This will happen when a client clicks on a board name from the list of existing board on the left of the GUI.  Expected behavior: The \texttt{User} associated with this client will be removed from the list of users working on the previous board.  It will then be added to the list of users working on the newly selected board.  This will be reflected on the list of users on the bottom left of the GUI for anyone working on the newly selected board.  In addition, the client who changed selections will have his drawing space refreshed to contain an up-to-date version of his/her newly selected board.  The client will then be able to modify this board as usual.

\subsubsection{Creating a New Board}
Tests behavior when a client attempts to create a new multi-user whiteboard.  This event will occur when the client clicks the "+" button located below the list of existing whiteboards.  When this button is clicked, the client will be prompted to enter a name for the new whiteboard. If the name is not valid (already in use), the server will assign a default one.  The new whiteboard will appear in the list of existing whiteboards in the GUI, so that all users are able to access it.  The client who created the whiteboard will automatically have his GUI switched so that he is viewing the newly created board, in accordance to the "Loading a Board" test above. The newly created \texttt{MasterBoard} will begin as a blank white canvas, with the standard 800x600 pixel dimensions.

\subsubsection{Deleting a Board}
Tests behavior when a client deletes an existing board. This will occur when a client clicks the "-" button located below the list of existing whiteboards.  To prevent clients from accidentally (or purposefully) deleting others' work, the "-" button will always delete the whiteboard that the client is currently viewing. In addition, it will be impossible to delete a whiteboard if only one is currently is existence, to prevents errors arising from a lack of whiteboards.  Deletion will completely remove the selected \texttt{MasterBoard} object from the server memory. It will be reflected in the list of whiteboards that appears in the user GUI, where the name of the deleted whiteboard will be removed for all users. Other clients who are working on a whiteboard when it is deleted will be notified via a pop-up message that their board has been deleted.  Their workspace will then not have any whiteboard selected, meaning that they will need to load another whiteboard before continuing to draw.  All requests sent to the server referencing the deleted whiteboard will be ignored after board deletion.

\subsubsection{Choosing Stroke Thickness}
This will test the user's ability to select a stroke thickness for drawing purposes. Thickness will be controlled through the \texttt{Width} component of a \texttt{Java 2D Stroke}. A variety of stroke widths will be available for selection from the thickness panel near the bottom left corner of the GUI. Each image in this area will be associated with a specified stroke thickness. Clicking on one of these images will switch the user's default stroke thickness to the width associated with the selected image.

\subsubsection{Choosing Stroke Color}
This will test the user's ability to select a drawing color. Thickness will be controlled through the \texttt{Color} component of \texttt{Java 2D Graphics}. A variety of colors will be available for selection from the color panel along the bottom of the GUI. Clicking on one of these given colors will switch the user's default stroke color to the selected color.

\subsubsection{"More Colors" Button}
In addition to choosing a color as listed above, users should be able to select from a wider range of colors in a \texttt{JColorChooser} window. This window will appear when a client clicks the "more colors" button to the right of the GUI's color panel. The client will then be able to change his default stroke color to any color provided in the \texttt{JColorChooser} window. If the user clicks outside of the \texttt{JColorChooser} window while it is open, the window will be closed and the client's color will not be changed.

\subsubsection{"Erase" Button}
The erase button will function as a toggle, switching the client between his previously chosen color and white. As the default board color is white, painting with white will appear visually the same as erasing an existing portion of the whiteboard. When the user selects "erase" for the first time, his/her color choice will switch to white.  Clicking the "erase" button a second time will switch the user back to whichever color was selected before "erase" was clicked the first time.

\subsubsection{"Clear" Button}
The "clear" button will completely erase the client's currently selected board.  On the server side, this will delete all strokes that have ever been applied to the selected whiteboard. As a result, the whiteboard will appear as it did when it was first created - blank white. This change, like any other painting operation, will be seen by all clients viewing the whiteboard.

\subsubsection{Drawing on a Whiteboard}

\paragraph{Whiteboard Selected}
When a client clicks and drags across the whiteboard area, a new instance of a \texttt{Stroke} object will be created to match the users' input with regards to length, shape, location, color, etc.  This \texttt{Stroke} will be sent to the server, where it will be associated with the whiteboard it should be applied to. The server will then update the views of all other users working on the same whiteboard to include the new stroke.  In this way, all users will be able to see all other users' changes in real time

\paragraph{No Whiteboard Selected}
If the client has not selected a whiteboard, attempting to interact with the canvas will result in a message box appearing, warning the user to select a whiteboard before drawing. The user's attempts to draw will produce no visible result on the canvas, nor will there be anything sent to the server.

\subsubsection{Concurrent Board Operations}

\paragraph{User A Deletes while User B Modifies}
The selected \texttt{MasterBoard} instance will be deleted. Both users (as well as any other users on the board) will be notified via message box that it has been deleted. They will all then need to select another whiteboard in order to continue drawing. User B's request to the server will be ignored.

\paragraph{Switch then Disconnect}
Test when a user switches boards, then immediately disconnects from the server. The user's name should disappear from all lists, including the board he/she switched to just before disconnecting. All other users should see this in the list of users currently editing whiteboards.

\paragraph{Rapid-Switching between Boards}
Test when a client switches very quickly between a number of boards.  The server should only attempt to update the client's view to the most recent whiteboard selected - that is, it should abandon any attempts to update the client's board to a previously selected board, even if the loading process is not complete. At the same time, the client's GUI should ignore any server messages regarding a whiteboard that is no longer selected. In this way, the client will only ever see their most recent whiteboard selection loading.

\paragraph{Edit then Switch}
Test when a user makes a stroke on a whiteboard, then very quickly switches to another board. The client's edits should still be sent to the server, and be reflected on the server's copy of that board, so that all other users can see the edit. The client who made the edit, however, should not have it reflected back to him/her by the server. Instead, the client should go through the normal process of loading their newly-selected board.

\subsubsection{User Disconnect}
Test the normal functionality of a user disconnecting from the server. The user's GUI client should close, and the user's name should be removed from the whiteboard it was currently working on, meaning that it should not appear anywhere on the server. All other clients should be able to see this change.

\end{document}
